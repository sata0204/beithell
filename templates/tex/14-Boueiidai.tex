\daigaku{(大学名)大学・(学部名)学部}
{試験日\quad  2013年11月3日\quad 時間100分\quad \suuichi\suuni\suusan\suua\suub (数列,ベクトル)\suuc (行列,曲線,確率)} %この行をどうするか考え中、会社側と相談します
\begin{Mgwaku}(4,10)[\textgt{}]
\begin{reidai}
\begin{shomonr}
\input{../../(年度)-(大学名)-(学部名)/(年度)-(大学名)-(学部名)-1-1/(年度)-(大学名)-(学部名)-toi-1-1}%%% 問題部分(ここを作る)
\end{shomonr}
\begin{shomonr}
\input{../../(年度)-(大学名)-(学部名)/(年度)-(大学名)-(学部名)-1-2/(年度)-(大学名)-(学部名)-toi-1-2}%%% 問題部分(ここを作る)
\end{shomonr}
\begin{shomonr}
\input{../../(年度)-(大学名)-(学部名)/(年度)-(大学名)-(学部名)-1-3/(年度)-(大学名)-(学部名)-toi-1-3}%%% 問題部分(ここを作る)
\end{shomonr}
\end{reidai}
\input{../../(年度)-(大学名)-(学部名)/(年度)-(大学名)-(学部名)-2/(年度)-(大学名)-(学部名)-toi-2}%%% 問題部分(ここを作る)
\input{../../(年度)-(大学名)-(学部名)/(年度)-(大学名)-(学部名)-3/(年度)-(大学名)-(学部名)-toi-3}%%% 問題部分(ここを作る)
\end{Mgwaku}

\begin{multicols}{2}
\input{../../(年度)-(大学名)-(学部名)/(年度)-(大学名)-(学部名)-1-1/(年度)-(大学名)-(学部名)-kai-1-1}%%% 解答部分(ここを作る)
\input{../../(年度)-(大学名)-(学部名)/(年度)-(大学名)-(学部名)-1-2/(年度)-(大学名)-(学部名)-kai-1-2}%%% 解答部分(ここを作る)
\input{../../(年度)-(大学名)-(学部名)/(年度)-(大学名)-(学部名)-1-3/(年度)-(大学名)-(学部名)-kai-1-3}%%% 解答部分(ここを作る)
\vspace{2mm} 
\input{../../(年度)-(大学名)-(学部名)/(年度)-(大学名)-(学部名)-2/(年度)-(大学名)-(学部名)-kai-2}%%% 解答部分(ここを作る)
\vspace{2mm} 
\input{../../(年度)-(大学名)-(学部名)/(年度)-(大学名)-(学部名)-3/(年度)-(大学名)-(学部名)-kai-3}%%% 解答部分(ここを作る)
\vspace{2mm} 
\begin{waku}
\input{../../(年度)-(大学名)-(学部名)/(年度)-(大学名)-(学部名)-end/(年度)-(大学名)-(学部名)-end}%%% 問題部分(ここを作る)
\end{waku}
\end{multicols}
